
\documentclass[a4paper, 10pt]{article}


\usepackage[polish]{babel}
\usepackage[utf8]{inputenc}
\usepackage[OT4]{fontenc}
\usepackage{geometry}
\usepackage{ulem}
\RequirePackage{url}


\setlength{\parindent}{0cm}
\setlength{\parskip}{3mm plus1mm minus1mm}

\geometry{verbose,a4paper,tmargin=2.4cm,bmargin=2.4cm,lmargin=2.4cm,rmargin=2.4cm}
\usepackage{graphicx} % wstawianie obrazkow


%%%%%%%%%%%%%%%%%%%%%%%%%%%%%%%%%%%%%%%%%%%%%%%%


\title{{\bf {Grafy i sieci}} \\ {\large Sprawozdanie 1}}
\date{\today}
\author{Filip Nabrdalik \\Agata Taraszkiewicz}

%%%%%%%%%%%%%%%%%%%%%%%%%%%%%%%%%%%%%%%%%%%%%%%%
\begin{document}
\bibliographystyle{alpha}
%%%%%%%
\null  % Empty line
\nointerlineskip  % No skip for prev line
\vfill
\let\snewpage \newpage
\let\newpage \relax
\maketitle 
\let \newpage \snewpage
\vfill
\break % page break
%%%%%%%%%%%%%%%%%%%%%%%%%%%%%

\tableofcontents

\newpage

\newcommand{\ang}[1]{(ang. {\em #1}\/)}
\newcommand{\e}[1]{{\em #1}\/}





\section{Treść zadania}

{\bf{Zadanie}}

Porównanie algorytmów znajdowania minimalnego drzewa rozpinającego.

Wybrać, zaimplementować i przebadać co najmniej dwa algorytmy znajdowania minimalnego drzewa rozpinającego w grafie nieskierowanym. 


\section{Opis}


Celem projektu jest implementacja oraz eksperymentalna ocena wydajności dwóch algorytmów znajdowania minimalnego drzewa rozpinającego \ang{MST, Minimum Spanning Tree} 
w grafie nieskierowanym. Do badań zostały wybrane drzy algorytmy: Kruskalla, Prima oraz Borůvki. Należą one do grupy algorytmów zachłannych tj. nie dokonujących oceny czy w kolejnych krokach 
jest sens wykonywać dane działanie, lecz podejmują decyzję lokalnie optymalną, czyli najlepszą dla danego kroku.
%Jeszcze moze byc i ten Boruvka algorithm: na wiki jest to ladnie wytlumaczone https://pl.wikipedia.org/wiki/Algorytm_Bor%C5%AFvki	


Eksperymenty badawcze będą polegały na ocenie czasu pracy oraz zużycia pamięci podczas wykonywania trzech algorytmów dla grafów o róznej strukturze, ilosci wierzchołków $|V|$ 
oraz krawedzi $|E|$ na komputerze klasy PC. 

Implementacja algorytmów oraz struktur danych zostanie zrealizowana w języku programowania C++ z naciskiem na minimalizacje zużycia pamięci operacyjnej
oraz czasu procesora. 






\subsection{}



%BIBLIOGRAFIA
\nocite{*}
\bibliography{bibliografia}


\end{document}


