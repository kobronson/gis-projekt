
\documentclass[a4paper, 10pt]{article}


\usepackage[polish]{babel}
\usepackage[utf8]{inputenc}
\usepackage[OT4]{fontenc}
\usepackage{geometry}
\usepackage{ulem}
\RequirePackage{url}


\setlength{\parindent}{0cm}
\setlength{\parskip}{3mm plus1mm minus1mm}

\geometry{verbose,a4paper,tmargin=2.4cm,bmargin=2.4cm,lmargin=2.4cm,rmargin=2.4cm}
\usepackage{graphicx} % wstawianie obrazkow


%%%%%%%%%%%%%%%%%%%%%%%%%%%%%%%%%%%%%%%%%%%%%%%%


\title{{\bf {Grafy i sieci}} \\ {\large Sprawozdanie 2}}
\date{\today}
\author{Filip Nabrdalik \\Agata Taraszkiewicz}

%%%%%%%%%%%%%%%%%%%%%%%%%%%%%%%%%%%%%%%%%%%%%%%%
\begin{document}
\bibliographystyle{alpha}
%%%%%%%

\maketitle 


%%%%%%%%%%%%%%%%%%%%%%%%%%%%%


\newcommand{\ang}[1]{(ang. {\em #1}\/)}
\newcommand{\e}[1]{{\em #1}\/}





\section{Treść zadania}

{\bf{Zadanie}}

Porównanie algorytmów znajdowania minimalnego drzewa rozpinającego.

Wybrać, zaimplementować i przebadać co najmniej dwa algorytmy znajdowania minimalnego drzewa rozpinającego w grafie nieskierowanym. 

 

\section{Opis badanych algorytmów}
	\subsection{Algorytm Prima}
	\subsection{Algorytm Kruskalla}
	\subsection{Algorytm Borůvki}
\section{Planowana złożoność obliczeniowa}
\section{Struktury danych}
\section{Założenia programu}
\section{Projekt testów}
	
	





%BIBLIOGRAFIA
\nocite{*}
\bibliography{bibliografia}


\end{document}


