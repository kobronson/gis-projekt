
\documentclass[a4paper, 10pt]{article}


\usepackage[polish]{babel}
\usepackage[utf8]{inputenc}
\usepackage[OT4]{fontenc}
\usepackage{geometry}
\usepackage{ulem}
\RequirePackage{url}


\setlength{\parindent}{0cm}
\setlength{\parskip}{3mm plus1mm minus1mm}

\geometry{verbose,a4paper,tmargin=2.4cm,bmargin=2.4cm,lmargin=2.4cm,rmargin=2.4cm}
\usepackage{graphicx} % wstawianie obrazkow


%%%%%%%%%%%%%%%%%%%%%%%%%%%%%%%%%%%%%%%%%%%%%%%%



%%%%%%%%%%%%%%%%%%%%%%%%%%%%%%%%%%%%%%%%%%%%%%%%
\begin{document}
%%%%%%%
%%%%%%%%%%%%%%%%%%%%%%%%%%%%%



\newcommand{\ang}[1]{(ang. {\em #1}\/)}
\newcommand{\e}[1]{{\em #1}\/}




\title{GIS: Sprawozdanie 1}
\author{Filip Nabrdalik\\ Agata Taraszkiewicz}
\date{} 
\maketitle 

\section{Treść zadania}

{\bf{Zadanie}}

Porównanie algorytmów znajdowania minimalnego drzewa rozpinającego.

{\it Wybrać, zaimplementować i przebadać co najmniej dwa algorytmy znajdowania minimalnego drzewa rozpinającego w grafie nieskierowanym.} 


\section{Opis}


Celem projektu jest zbadanie algorytmów znajdowania minimalnego drzewa rozpinającego (ang. MST, Minimum Spanning Tree). Minimalne drzewo rozpinające grafu jest drzewem rozpinającym o najmniejszej sumie wag krawędzi. 
Do pracy badawczej wybraliśmy trzy algorytmy rozwiązujące problem poszukiwania MST w grafie nieskierowanym:
\begin{itemize}
\item Kruskalla,
\item Prima,
\item Borůvki
\end{itemize}
Algorytmy należą do grupy algorytmów zachłannych, czyli wybierających decyzje lokalnie optymalną w danym kroku. Złożoność obliczeniowa powyższych algorytmów jest liniowo-logarytmiczna.

Badanie wszystkich algorytmów będzie polegało na ocenie wydajności oraz niezawodności w zależności od dostarczonych danych testowych. Do testów zostaną wykorzystane grafy nieskierowane o różnej strukturze, ilości wierzchołków $|V|$ oraz krawędzi $|E|$. Badanie wydajności algorytmów będzie polegało na pomiarze czasu ich wykonania oraz zużycia pamięci. Badanie niezawodności będzie obejmowało próbę takiego dobrania danych testowych spełniających warunki zadania, aby w rozsądnym czasie algorytm nie znalazł minimalnego drzewa rozpinającego.

Implementacja algorytmów oraz struktur danych zostanie zrealizowana w języku programowania C++ z naciskiem na minimalizacje zużycia pamięci operacyjnej oraz czasu procesora. Do testów zostanie wykorzystany komputer klasy PC lub maszyna wirtualna z systemem Linux. Dalsze decyzje projektowe zostaną podjęte w następnej fazie.


\section{Literatura}

1. Ninna Lei, Three minimum spanning tree algorithms, \url{http://homes.cs.washington.edu/~jinna/ugrad_work/thesis.pdf }
 
2.  Igor Podsechin, Comparing minimum spanning tree algorithms, \url{http://www.aka.fi/Tiedostot/Tiedostot/Viksu/Viksu%202008/IgorPodsechin.pdf}

3. Joseph. B. Kruskal, On the Shortest Spanning Subtree of a Graph and the Traveling Salesman Problem, \url{www.cmat.edu.uy/~marclan/TAG/Sellanes/Kruskal.pdf }

4. Algorytmy Grafowe \url{http://www.algorytm.org/algorytmy-grafowe/algorytm-prima.html}
\end{document}


