
\documentclass[a4paper, 10pt]{article}

%polskie znaki
\usepackage[polish]{babel}
\usepackage[utf8]{inputenc}
\usepackage[OT4]{fontenc}


\usepackage{geometry}
\usepackage{ulem}
\RequirePackage{url}


\setlength{\parindent}{0cm}
\setlength{\parskip}{3mm plus1mm minus1mm}

\geometry{verbose,a4paper,tmargin=2.4cm,bmargin=2.4cm,lmargin=2.4cm,rmargin=2.4cm}
\usepackage{graphicx} % wstawianie obrazkow


%%%%%%%%%%%%%%%%%%%%%%%%%%%%%%%%%%%%%%%%%%%%%%%%


\title{{\bf {Grafy i sieci}} \\ {\large Sprawozdanie 1}}
\date{\today}
\author{Filip Nabrdalik \\Agata Taraszkiewicz}

%%%%%%%%%%%%%%%%%%%%%%%%%%%%%%%%%%%%%%%%%%%%%%%%
\begin{document}
\bibliographystyle{alpha}
%%%%%%%
\null  % Empty line
\nointerlineskip  % No skip for prev line
\vfill
\let\snewpage \newpage
\let\newpage \relax
\maketitle 
\let \newpage \snewpage
\vfill
\break % page break
%%%%%%%%%%%%%%%%%%%%%%%%%%%%%

\tableofcontents

\newpage






\section{Treść zadania}

{\bf{Zadanie}}

Porównanie algorytmów znajdowania minimalnego drzewa rozpinającego.

{\bf{Opis}}

Wybrać, zaimplementować i przebadać co najmniej dwa algorytmy znajdowania minimalnego drzewa rozpinającego w grafie nieskierowanym. 



\section{}




\section{}


\section{}
\subsection{}



%BIBLIOGRAFIA
\nocite{*}
\bibliography{bibliografia}


\end{document}


